\documentclass[11pt,a4paper]{article}
\usepackage{latexsym}
\usepackage{graphicx}
\usepackage{amssymb,amsmath}
\usepackage{mathtools,braket}
\usepackage[english]{babel}

% Parametri di stampa
\setlength{\topmargin}{-2.5cm} \setlength{\oddsidemargin}{0.3cm}
\setlength{\evensidemargin}{0.3cm}
%
\textheight=23.0truecm \textwidth=16.0truecm
%
\headheight=1.0cm \headsep=1.0cm
%
\renewcommand{\baselinestretch}{1.1}
\setlength{\unitlength}{1mm}

\parindent=6pt

%%%%%%%%%%%%%%%%%%%%%%%%%%%%%%%%%%%%%%%%%%%%%%%%%%%%%%%%%%%%%%%%%%%%%%%%%%%%

\newcommand{\be}{\begin{equation}}
\newcommand{\ee}{\end{equation}}
\newcommand{\bea}{\begin{eqnarray}}
\newcommand{\eea}{\end{eqnarray}}
\newcommand{\al}{\alpha}
\newcommand{\br}{\bar{\rho}}
\newcommand{\bg}{\bar{g}}
\newcommand{\bpi}{\bar{\pi}}

\newcommand{\op}[1]{\hat {#1}}

%%%%%%%%%%%%%%%%%%%%%%%%%%%%%%%%%%%%%%%%%%%%%%%%%%%%%%%%%%%%%%%%%%%%%%%%%%%%

\begin{document}

%%%%%%%%%%%%%%%%%%%%%%%%%%%%%%%%%%%%%%%%%%%%%%%%%%%%%%%%%%%%%%%%%%%%%%%%%%%%

List of major changes apported to the paper:
\begin{itemize}
 \item Much efforts have been spent in trying to clarify better the relation between localization properties and features of the computational basis needed 
 to provide precise results. This topic has been discussed in all the text. Major modifications under this perspective have been done in the introduction, in section II at pag. 5 before the analysis of fig. 2,in section IV and in the conclusions. 
 \item The introduction has been deeply revised. In particular we have reduced the length of the first two paragraph concerning the differences between accuracy and precision.
 \item We added a comment after eq. (7) on the non-hermiticity of the response density and corrected the equation (11).
 \item Added a dedicated subsection (actually the IIA at page 4)  to discuss the localization properties of the FS states for static perturbations. 
 \item Added a panel in figure 2 that contain the plot of the real part of $\alpha(\omega)$ in calculated in different computational domain. This new plot aims to evidence that the localization
 properties of th FS affect, on equal footing, both the real and the imaginary part of the linear response functional.
 \item We have reorganized the material after section III A. In particular we have introduced a new section (number IV in the actual version of the paper) in which we have discussed both the physical features of the excitation spectrum and the relative importance of localized vs. delocalized sector in expressing the susceptibility functional. This section also collects computational results for molecules beyond the case of $CO$ and benzene, already discussed in the previous version of the paper.  
\end{itemize}

\section*{Response to the first referee}

We wish to thank the referee for the positive review on the paper. The text has been deeply revised according to the suggestions of both the referees.

Here we report some detailed answer concerning the main aspects raised in the referee's report:
\begin{enumerate}
  \item \emph{on the point concerning the dimension of the simulation box needed to achieve the convergence of the delocalized states of the $CO$ molecule}. 
  
  Delocalized fluctuation states do not achieve convergence as long as the simulation box is increased. Since they behave as continuum states they suffer of the same continuum collapse behavior described in section IA for the unbound solution of the Schr\"odinger equation. What is expected to converge is the spectral function, or declined in the present case the real and imaginary part of the dynamical polarizability. 
  The value of convergence is system/observable and perturbation dependent. It can be found through an explicit convergence analysis by introducing an acceptance tolerance, however this procedure is out of the scope of our analysis and has not been performed. 
  
  Nonetheless, to have an idea of the dimension of the box needed to achieve convergence we can refer to the case of the Benzene molecule and make usage
  of the result found in our analysis, compared to the one reporte in fig. 2 of \cite{baroni2008}. Within our analysis we show that a box linear size of about 18 Angostroem is sufficient to produce converged results for excitation spectrum below IP (around 6.5 eV in the present case). On the contrary, results of \cite{baroni2008} evidence that to achieve convergence up to 30 eV a box of linear dimension of more than 31 Angstroem is needed, so going from the localized to the delocalized sector of excitation spectrum requires a relevant increase in the simulation domain. 
  
  \item \emph{on the point concerning the convergence rate of the delocalized vs. localized states with respect
  to the grid spacing}
  
 In presenting the results of this paper care has been taken to ensures the convergence of the results wrt the computational parameters. We have found that there are no relevant difference in the convergence rate of localized vs delocalized quantities (for instance excitations) in function of the grid spacing. This is due to the fact that localized and delocalized states differ in their asymptotic properties whereas the grid spacing convergence is influenced by the fast short-range oscillation of the functions.
 
 Add a comment in the text somewhere?
 
 The values of the grid spacing used for the calculations have been added in appendix C.
 
\item Other minor corrections:
\begin{itemize}
 \item The insets of fig. 4 and 5 have been increased.
 \item the \emph{reasonable behavior of the perturbed potential} at the end of section IIIA has been clarified. It simply means that the
 the perturbed potential $\op V'[\op E]$ gives rise to a bound state when applied to $\ket{\psi_p}$.
 \item explicit reference to the inset of fig. 5 has been performed in the text.
\end{itemize}

\end{enumerate}

\section*{Response to the second referee}

We thank the referee for its comment on the paper. Thank to his/her advice and criticism we have tried to clarify our message. 
We try to answer to the critical points raised out by the review:

\begin{enumerate}
 \item \emph{on the relevance of the symmetry of the molecule in determining the delocalized vs localized character of the an excitation}
 
 In the paper we have introduced a the dyadic representation of excitation $\op E_a$ operator (eq. 16) and we have presented an argument to
 demonstrate that the associated excited states $\phi_p^a,\chi_p^a$ are localized only if the condition $\Omega_a<\epsilon_p$ is satisfied. 
 The set of $p$ labels relevant for the construction of a specific excitation depends of the symmetry...
 
 %The structure of eq. 16 states that when both localized and delocalized excited states partecipate to the excitation 
 
 \item \emph{regarding the localized excitations which can still be found above the IP threshold (Fig 5a), is there any scope to target these, to the exclusion of other states so that a set of just the localizedexcitations can be calculated at reduced computational cost compared to the whole set?}

 
 Answer using the results for the statical polarizability described in section III.C...
 
 \item \emph{exclusion ''by design'' of the delocalized part of the excitation spectrum}
 
 On the base of what written by the referee it seems to me that this sentence could be interpreted in two different way. 
 
 The first one concerns with the removal of the contribution of the delocalized states in a specific excitation. This  
 
 The second one indicate what happens when one uses a basis able to express localized states only...
 
 
 
\end{enumerate}

\bibliographystyle{elsarticle-num}
\bibliography{Analytic_biblio}

%%%%%%%%%%%%%%%%%%%%%%%%%%%%%%%%%%%%%%%%%%%%%%%%%%%%%%%%%%%%%%%%%%%%%%%%%%%%

\end{document}

%%%%%%%%%%%%%%%%%%%%%%%%%%%%%%%%%%%%%%%%%%%%%%%%%%%%%%%%%%%%%%%%%%%%%%%%%%%%
