\documentclass[reprint,aps,prb]{revtex4-1}

%% Language and font encodings
\usepackage[english]{babel}
%\usepackage[utf8x]{inputenc}
%\usepackage[T1]{fontenc}

%% Useful packages
\usepackage{graphicx}
\usepackage{amsmath,amsfonts,amssymb,amsthm}
\usepackage{mathtools,braket}
\usepackage{xifthen}

% notation for standard math
\renewcommand{\d}{\partial}
\newcommand{\half}{\frac{1}{2}}
\newcommand{\dd}{{\rm d}}
\renewcommand{\r}{{\bf r}}
\newcommand{\br}{\bar{\bf r}}
\newcommand{\x}{{\bf x}}
\newcommand{\eps}{\epsilon}
\newcommand{\bomega}{\bar\omega}
\newcommand{\bbomega}{\bar{\bomega}}
\newcommand{\ii}{\mathrm{i}}
\newcommand{\intdef}[3]{\int_{#1}^{#2} \dd {#3}}
\newcommand{\intover}[1]{\int_{-\infty}^{+\infty} \dd {#1}}
\newcommand{\sint}{\mathrlap{\displaystyle\int}
\mathrlap{\textstyle\sum}
\phantom{\mathrlap{\displaystyle
\int}\textstyle\sum}}

% notatiotion for equation environments
\newcommand{\be}{\begin{equation}}
\newcommand{\ee}{\end{equation}}
\newcommand{\ba}{\begin{eqnarray}}
\newcommand{\ea}{\end{eqnarray}}
\newcommand{\baa}{\begin{align}}
\newcommand{\eaa}{\end{align}}
\newcommand{\nn}{\notag}
\newcommand{\qq}{\qquad}
\newcommand{\lb}{\label}
\newcommand{\mat}[1]{\begin{pmatrix} #1\end{pmatrix}}

% notation for the operators
\newcommand{\op}[1]{\hat {#1}}
\newcommand{\sop}[1]{\op{\op {#1}}}
\newcommand{\commutator}[2]{\left[ {#1} , {#2} \right]}
\newcommand{\trace}[1]{\mathrm{tr}\left(#1\right)}
\newcommand{\argument}[1]{\ifthenelse{\isempty{#1}{}}{}{(#1)}}
\newcommand{\matop}[1]{\mathbf{#1}}
\newcommand{\optr}[1]{\check #1}
\newcommand{\opskew}[1]{{\op {#1}}_{\perp}}

% notation for the states
\newcommand{\tket}[1]{| \tilde #1 \rangle}
\newcommand{\tbra}[1]{\langle \tilde #1 |}
\newcommand{\brket}[2]{\langle  #1 | #2 \rangle} %standard braket
\newcommand{\tbraket}[2]{\langle \tilde #1 | #2 \rangle}
\newcommand{\ketbra}[2]{| #1 \rangle \langle #2 |}
\newcommand{\tketbra}[2]{| #1 \rangle \langle \tilde #2 |}
\newcommand{\sket}[2]{| #2)^{#1}}
\newcommand{\sbra}[2]{( #2|_{#1}}
\newcommand{\sketor}[2]{| #2]^{#1}}
\newcommand{\sbraor}[2]{[ #2|_{#1}}
\newcommand{\sbraket}[2]{\braket{\op{#1} | \op{#2}}}
\newcommand{\dket}[1]{\Bigl| #1 \Bigr)}
\newcommand{\dbra}[1]{\Bigl(#1 \Bigr|}
\newcommand{\dbraket}[2]{\Bigl(#1 \Bigl| #2 \Bigr)}
\newcommand{\dketor}[1]{\Bigl| #1 \Bigr]}
\newcommand{\dbraor}[1]{\Bigl[#1 \Bigr|}
\newcommand{\hket}[1]{| #1 ]}
\newcommand{\hbra}[1]{[ #1 |}
\newcommand{\hbraket}[2]{[#1 | #2 ]}

% special operators
\newcommand{\dmnot}{\op{\rho}_0}
\newcommand{\dm}{\op{\rho}}
\newcommand{\hnot}{\op{H}_0}
\newcommand{\hone}[1]{\op{H}_1\argument{#1}}
\newcommand{\transition}[1]{\op T_{#1}}
\newcommand{\excite}[2]{\op e_{{#1}{#2}}}
\newcommand{\decay}[2]{\op d_{{#1}{#2}}}
\newcommand{\Liouv}{\sop{\mathcal L}}
\newcommand{\Liouvnot}{\sop{\mathcal L_0}}
\newcommand{\coupl}{\sop{\mathcal K}}
\newcommand{\honetmp}[1][]{\op{H_1}\argument{#1}} 
\newcommand{\identity}{\op{\mathbb I}}
\newcommand{\rmat}[1]{\optr R}


\begin{document}

%\preprint{APS/123-QED}

\title{On the analytic structure of the linear response susceptibility in open systems}

\author{Luigi Genovese}
\affiliation{Laboratoire de Simulation Atomistique (LSim), SP2M, INAC, CEA-UJF, 17 Av. des Martyrs,
38054 Grenoble, France}
\author{Marco D'Alessandro}%
\affiliation{Istituto di Struttura della Materia-CNR (ISM-CNR), Via del Fosso del Cavaliere 100, 00133 Roma, Italia}

\date{\today}

\begin{abstract}
We analyze how the fundamental equations which govern the linear response properties of a quantum system are influenced by the presence of open boundaries.
In particular we focus our attention on linear-response (LR)-TDDFT calculation of molecular systems. We show that optical excitations of energy below the LR ionization potential 
behave as observable, localized quantities, that can be computationally calculated with high precision, provided an adequate level of completeness.
Above this energy threshold, the analytic structure of the susceptibility functional might be separated in two sectors: one can still be described in terms of observable excitation 
with discrete energies, that can be captured within computational setups tailored for localized states. However such excitations are embedded in a continuum of excited states 
that, on the contrary, behave as plane waves and cannot be described efficiently by the same computational techniques. Such excitations do not exhibit observable features and 
\emph{cannot} be considered as poles of the dynamical polarizability. Nonetheless, the contribution of such continuum of excitation reveals to be of paramount importance even 
for expressing LR quantities in the infrared regime like for example static polarizability.
This is particularly evident for small molecules; we show how these considerations might be influenced by the size of molecular system.
We then present indicators that can help to quantify such potential observable property of an excitation, that can be evaluated 
in any discretization scheme.  This result is a inherent behavior of the system's Liouvillian, and does not depend on computational treatment of the 
unoccupied subspace.
\end{abstract}

%\pacs{Valid PACS appear here}
\maketitle

\section{Introduction}

.....

\section{Theory}

Linear response can be effectively applied to the analysis of the electronic excitations in molecular and condensed-matter systems. In this framework one consider a system 
described by the Hamiltonian $\hnot$ and assume that the ground-state density $\dmnot$ has been provided. Linear response formalism allows us to evaluate the modification 
of the expectation value of a generic observable induced by a time-dependent perturbing field acting on the system. This quantity, written in the frequency domain, can be expressed 
through the evaluation of the \emph{linear response functional}:
\be\lb{LinearResponseFunctDef1}
\braket{\delta\op O}(\omega) = \trace{\dm'(\omega)\op O} \;,
\ee
where the operator $\op O$ represents the observable\footnote{For a sake of concreteness, we are limiting our considerations to observables that do not explicitly depend on time} 
under inspection. The modification of the ground-state density induced by the perturbation is expressed by the \emph{response density operator} $\dm'(\omega)$ that, in the linear 
response regime, is a linear function of the perturbing field $\op\Phi(\omega)$. For this reason it is possible to define the \emph{susceptibility} superoperator, as follows:
\be\lb{SusceptibilityDef1}
\dm'(\omega) = \sop{{\cal R}}(\omega) \op\Phi(\omega) \;. 
\ee
Susceptibility describes an intrinsic feature of the unperturbed system and determines its response to \emph{any} perturbations, in the linear response regime.
%ndependent from the specific perturbation that acts on the system and codifies the fundamental aspects which determine the response of the system. % is the fundamental object at the basis of 

The analysis presented in this section is devoted to a critical review of the fundamental equations which define the response density operator in the linear response regime.    
The aim of this process is to....

%the construction of two different representations of the response density operator... 

\subsection{Fluctuation states. A (de-)localization argument for the response density}

The response density operator is expressed as the first-order variation of ground state density $\sum_{\{p\}} \ket{\psi_p} \bra{\psi_p}$,
%\be\label{rhodef1}
%\dmnot = \sum_{\{p\}} \ket{\psi_p} \bra{\psi_p}  \;,
%\ee
where the set $\{\ket{\psi_p}\}$ denote the occupied states, \emph{i.e} the subset of the bound eigenstates of $\hnot$ with energy $\eps_p$ lower than the Fermi level.
Using this notation we denote the energy of the HOMO level as $\eps_h$ and identify the value of the first ionization potential with $|\eps_h|$. 
In what follows we will assume that the set of states $\{\ket{\psi_p}\}$ are real functions, when project in the $\r$-representation.  

By considering the linear order time evolution of the eigenstates of $\hnot$ in presence of the perturbation we can derive the equation of motion of the response density in the form of 
a quantum Liouville equation (for example see \cite{baroni2008}). This procedure suggests us to introduce an explicit representation of the response density in terms of a set of 
$\omega$-dependent \emph{fluctuation states} $\{\ket{f_p(\omega)}\}$, as follows: 
\be\lb{rhoPrimeFluctuationStateDef1}
\dm'(\omega) = \sum_p\left(\ketbra{\psi_p}{f_p(-\omega)} + \ketbra{f_p(\omega)}{\psi_p}\right) \;.
\ee
Fluctuation states are, by construction, orthogonal to the occupied eigenstates of $\hnot$. Moreover, an analysis of the Liouville equation for the response density written in this fashion
evidences that their equations of motion are written in the form of a modified Sternheimer equation \cite{mahan1980}:
%An analysis of the Liouville equation for the response density written in this fashion evidences that fluctuation states are orthogonal to the occupied eigenstates 
%of $\hnot$ and their equations of motion are written in the form of a modified Sternheimer equation \cite{mahan1980}:
\be\lb{fluctuationStateEqMotion1}
\left[\omega - (\hnot-\eps_p)\right]\ket{f_p(\omega)} = \op Q_0(\op\Phi(\omega)+\op V'(\omega))\ket{\psi_p} \;,
\ee
where $\op Q_0=\identity-\dmnot$ is the projector in the subspace orthogonal to the occupied states of $\hnot$ and the operator $\op V'(\omega)$ is a perturbing term due to the density-dependence 
of the unperturbed Hamiltonian. %$\hnot$. %the unperturbed Hamiltonian. 
%and describes the perturbation induced by the variation of $\hnot$ under the action of the perturbing field.
%The  r.h.s of \eqref{fluctuationStateEqMotion1} expresses the complete perturbing potential

The interesting aspect of equations \eqref{fluctuationStateEqMotion1} in the perspective of the present analysis is that they allow us to present a 
\emph{localization argument} for the fluctuation states. To clarify this statement we split the unperturbed Hamiltonian as the sum of the kinetic $\op T$ and potential 
$\op V$ term and solve \eqref{fluctuationStateEqMotion1} in the form of a self-consistent inhomogeneous Helmholtz equation by writing:
\be
\ket{f_p(\omega)} = \op G_H(\omega+\epsilon_p)\ket{s(\omega)} \;,
\ee
where we have introduced the Green function of the Helmholtz operator $\op G_H(\omega) = (\omega-\op T)^{-1}$, % with $k^2_p(\omega) = (\omega+\eps_p)$,
and the \emph{source term} is given by:
\be
\ket{s(\omega)} =  \op V\ket{f_p(\omega)} + \op Q_0(\op V'(\omega) + \op \Phi(\omega))\ket{\psi_p} \;.\nn
\ee
For open boundary conditions in three dimensions, the kernel of $\op G_H(\omega)$ in the coordinate representation reads: 
% \be\lb{HelmholtzKernelDef1}
% \bra{\r}\op G_H(\omega)\ket{\r'} = \frac{1}{4\pi} \left\{ 
% \begin{matrix}
% \frac{e^{-2k_p(\omega)|\r-\r'|}}{|\r-\r'|} \;,\; {\textrm for} \; \omega \leq |\eps_p| \\ 
% \frac{e^{\ii k_p(\omega)|\r-\r'|}}{|\r-\r'|} \;,\; {\textrm for} \; \omega > |\eps_p|
% \end{matrix}
% \right. \;,
% \ee
\be\lb{HelmholtzKernelDef1}
\bra{\r} \op G_H(\omega) \ket{\r'} = \frac{1}{4\pi} \begin{cases}
\frac{e^{2k|\r - \r' |}}{|\r- \r' |} \;,\; & {\textrm for}\; k^2 = \omega  < 0\; \\ 
\frac{e^{2\ii k |\r-\r' |}}{|\r-\r' |} \;, & {\textrm for} \; k^2 = \omega > 0
\end{cases} \;,
\ee
and the $\r$-representation of the fluctuation states $f_p(\omega,\r)$ is expressed through a convolution integral between the kernel \eqref{HelmholtzKernelDef1} and
$\brket{\r'}{s(\omega)}$.
An inspection of the structure of \eqref{HelmholtzKernelDef1} evidences that, for each value of $p$, there exist a threshold value $\omega = |\eps_p|$, 
which govern the localization properties of the corresponding fluctuation state. Indeed, for $\omega > |\eps_p|$,  $\op G_H(\omega)$ behaves as a spherical wave and its
convolution with the source term always give raises to a $f_p(\omega,\r)$ delocalized in the whole accessible space. On the contrary, for values of $\omega$ below the threshold,
the kernel contains an exponential damping factor and %the localization features of $f_p(\omega,\r)$ are determined by the spatial behavior of the source term.  
$f_p(\omega,\r)$ exhibit a bound-state behavior in the long range.   

% \vspace{0.3cm}
% \noindent{TODO: Add comment to support the localization below threshold}. Two possible way. First way, use the fact that the $\ket{f_p(0)}$ is the nothing but the $p$-\emph{th} 
% eigenstate of the perturbed Hamiltonian with perturbing field $\op\Phi(0)$. The localization of $\ket{f_p(0)}$ can be explicitly checked by solving the ground-state problem for
% this Hamiltonian. How to extended to finite values of $\omega$ below threshold? Second way. Try and iterative solution of the self-consistent Helmholtz equation (below threshold) 
% starting from a localized guess for the fluctuation state. Then state something about the the fact that is reasonable to believe that (for a proper choice of the perturbing field)
% the operators appearing in the source term keep its support compact. If we are able to state this in a appropriate way the localization argument below threshold hold. In my opinion
% this way is better.
% \vspace{0.3cm}

% The knowledge of the long range behavior of the fluctuation states allows us to perform an assessment on the reliability of the evaluation of the linear response functional 
% \eqref{LinearResponseFunctDef1}. To elaborate this statement we make usage of the fluctuation states representation of the response density \eqref{rhoPrimeFluctuationStateDef1}, 
% so that:
% The knowledge of the long range behavior of the fluctuation states allows us to assess the spatial localization properties of the basic objects relevant in the contest of the 
% linear response. In particular, making usage of the fluctuation states representation of the response density \eqref{rhoPrimeFluctuationStateDef1}, we can express the linear 
% response functional \eqref{LinearResponseFunctDef1} as follows: 
This analysis allows us to comprehend that the spatial localization features of the building blocks relevant for the linear response depend critically on $\omega$. 
Indeed, the fluctuation states representation of the response density \eqref{rhoPrimeFluctuationStateDef1} implies that
\be\lb{LinearResponseFunctDef2}
\braket{\delta\op O}(\omega) = 2\sum_p \bra{\psi_p}\op O\ket{f_p(\omega)} \;,
\ee
so we can identify two different regimes, depending on the value of $\omega$. In the first one, realized for $\omega<|\eps_h|$, all the terms contributing to \eqref{LinearResponseFunctDef2} 
are expressed in terms of genuinely localized wave functions, while in the second one, for $\omega>|\eps_h|$, fluctuation states start behaving as unbound wave functions. 
%and the scalar products appearing in \eqref{LinearResponseFunctDef2} are only regularized by the long range behavior of the $\ket{\psi_p}$.
This fact implies that a computational setup has different chances of providing a reliable assessment of the linear response functional \eqref{LinearResponseFunctDef2} depending
to the value of $\omega$. 
This observation has relevant consequence on the capability of a computational setup to provide a reliable assessment of the linear response functional....  



\vspace{0.3cm}
\noindent{TODO: Numerical example (convergence below threshold of the Benzene spectrum computed in different computational setup).}


\subsection{Excitation operators. A spectral decomposition representation for the response density}

Inizio a mettere le formule da usare...forse...

Liouville equation:
\be\lb{LiouvillianRhopomegaDef1}
\left(\omega - \Liouv\right) \dm'(\omega) =  \commutator{\op\Phi(\omega)}{\dmnot}\;,\;\; 
\Liouv = \Liouvnot + \coupl \;.
\ee
% by introducing the \emph{coupling}
% superoperator:
% \be\lb{couplingSopDef1}
% \coupl = 
% \int \dd \r \dd \r' \commutator{\frac{\delta\hnot}{\delta \rho_0(\r,\r')}}{\dmnot}
% \trace{\cdot \ket{\r} \bra{\r'}} \;,
% \ee
Excitation operators as eigenstates of the Liouvillian:
\be
\Liouv \op E = E \op E \;, \qq
 \op{\tilde E} \Liouv = E \op{\tilde E}
\ee

\subsection{Comments. The analytic structure of the linear response susceptibility}

...........



%%%%%%%%%%%%%%%%%%%%%%%%%%%%%%%%%%%%%%%%%%%%%%%%%%%%%%%%%%%%%%%%%%%%%%%%%%%%%%%%%%%%%%%%%%%%%%%%%

% In the framework of the linear response formalism one is typically interested in computing the \emph{linear response functional}:
% \be\lb{LinearResponseFunctDef1}
% {\cal R}_{\op O}(\omega) = \trace{\dm'(\omega)\op O} \;,
% \ee
% that represents the perturbation-induced modification of the expectation value of an observable $\op O$\footnote{For a sake of concreteness, we are limiting our considerations 
% to observables that do not explicitly depend on time}. Here $\dm'(\omega)$ is the \emph{response density operator} that describes the $\omega$-dependent modification of the density 
% operator induced by the perturbation.
% 
% \subsection{Basic equations for the linear response formalism for electronic excitations}
% 
% The linear response formalism can be effectively applied to the analysis of the electronic excitations of molecular and condensed-matter systems, described by the (density-dependent) 
% ground-state hermitian Hamiltonian $\hnot$, coupled to an external perturbation described by the perturbing field $\op\Phi(\omega)$. The $\omega$ structure of $\op\Phi(\omega)$ codifies both 
% the intrinsic time dependence of the perturbation and the switching-on procedure, \emph{e.g.} adiabatic or sudden limit protocols. 
% 
% The response density is expressed as the first-order variation of ground state density
% \be\label{rhodef1}
% \dmnot = \sum_{\{p\}} \ket{\psi_p} \bra{\psi_p}  \;,
% \ee
% where $\{\psi_p\}$ denote the occupied states, \emph{i.e} the subset of the bound eigenstates of $\hnot$ with energy lower than the Fermi level.
% By considering the linear order time evolution of these states in presence of the perturbation we can express the response density in the $\omega$ domain as follows:
% \be\lb{rhoPrimeOmegaDef1}
% \dm'(\omega) = \ii \commutator{\dmnot}{\op A(\omega)} \;,
% \ee
% where $\op A(\omega)$ is a sum of two terms, both written as a triple convolution with the retarded/advanced Green functions $\op G_0^{\pm}$ of the unperturbed Hamiltonian:
% \be
% \op A(\omega) = \frac{1}{2\pi} \int_{-\infty}^\infty  \dd \omega'
% \op G_0^+(\omega') \left(\op \Phi(\omega)+\op V'(\omega)\right) \op G_0^-(\omega'-\omega)\;. \nn
% \ee
% The presence of the second addend is due to the density dependence of the unperturbed Hamiltonian and describes the perturbation induced by the variation of $\hnot$ under the
% action of the perturbing field. In the linear response regime this quantity can be expressed as the functional derivatives of $\hnot$ with respect to the density kernel 
% (evaluated at $\dm=\dmnot$) times the value of the induced density modification, that is:
% \be\label{VpDef1}
% \op V'(\omega)= 
% \int \dd \r \dd \r' \frac{\delta\hnot}{\delta \rho_0(\r,\r')}
% \trace{\dm'(\omega) \ket{\r} \bra{\r'}} \;.
% \ee
% 
% \subsection{Fluctuation states. A localization argument for the linear response functional}
% 
% We start this analysis by deriving an ulterior expression of the response density operator in the frequency space. We consider equation \eqref{rhoPrimeOmegaDef1} and observe that, for any 
% operator $\op O$, we may write:  
% \be
% \commutator{\dmnot}{\op O} = \dmnot \op O \op Q_0 - \op Q_0 \op O \dmnot\;,
% \ee
% where $\op Q_0= \identity - \dmnot$ is the projection operator in the subspace orthogonal to $\dmnot$. So, thanks to this relation we have that:
% \be
% \dm'(\omega) = \ii\sum_p\left(\ketbra{\psi_p}{\psi_p}\op A(\omega)\op Q_0 - \op Q_0\op A(\omega)\ketbra{\psi_p}{\psi_p}\right) \;.
% \ee
% This equation suggests us to define the \emph{fluctuation states}:
% \be\lb{fluctuationStateDef1}
% \ket{f_p(\omega)} \doteq -\ii\op Q_0\op A(\omega)\ket{\psi_p} \;, \quad
% \bra{f_p(\omega)} \doteq \bra{\psi_p}\ii\op A(-\omega)\op Q_0 \;,
% \ee
% where the $\omega$ dependence of the bra is consistent with hermitian conjugation property of the Fourier transform of a hermitian operator. Fluctuation states are, by 
% definition, orthogonal to the occupied eigenstates of $\hnot$, so that:
% \be
% \brket{f_p(\omega)}{\psi_q} = \brket{\psi_q}{f_p(\omega)} = 0 \;, \forall \, p,q,\omega \;.
% \ee
% The response density written in terms of the fluctuation states reads:
% \be\lb{rhoPrimeFluctuationStateDef1}
% \dm'(\omega) = \sum_p\left(\ketbra{\psi_p}{f_p(-\omega)} + \ketbra{f_p(\omega)}{\psi_p}\right) \;,
% \ee
% and consequently the linear response functional can be expressed as:
% \be\lb{LinearResponseFunctDef2}
% {\cal R}_{\op O}(\omega) = 
% \sum_p\left(\bra{f_p(-\omega)}\op O\ket{\psi_p} + \bra{\psi_p}\op O\ket{f_p(\omega)}\right) \;.
% \ee


%%%%%%%%%%%%%%%%%%%%%%%%%%%%%%%%%%%%%%%%%%%%%%%%%%%%%%%%%%%%%%%%%%%%%%%%%%%%%%%%%%%%%%%%%%%%%%%%%%





\bibliography{Analytic_biblio}

\end{document}
