\documentclass[reprint,aps,prb]{revtex4-1}

%% Language and font encodings
\usepackage[english]{babel}
%\usepackage[utf8x]{inputenc}
%\usepackage[T1]{fontenc}

%% Useful packages
\usepackage{graphicx}
\usepackage{amsmath,amsfonts,amssymb,amsthm}
\usepackage{mathtools,braket}
\usepackage{xifthen}

% notation for standard math
\renewcommand{\d}{\partial}
\newcommand{\half}{\frac{1}{2}}
\newcommand{\dd}{{\rm d}}
\renewcommand{\r}{{\bf r}}
\newcommand{\br}{\bar{\bf r}}
\newcommand{\x}{{\bf x}}
\newcommand{\eps}{\epsilon}
\newcommand{\bomega}{\bar\omega}
\newcommand{\bbomega}{\bar{\bomega}}
\newcommand{\ii}{\mathrm{i}}
\newcommand{\intdef}[3]{\int_{#1}^{#2} \dd {#3}}
\newcommand{\intover}[1]{\int_{-\infty}^{+\infty} \dd {#1}}
\newcommand{\sint}{\mathrlap{\displaystyle\int}
\mathrlap{\textstyle\sum}
\phantom{\mathrlap{\displaystyle
\int}\textstyle\sum}}

% notatiotion for equation environments
\newcommand{\be}{\begin{equation}}
\newcommand{\ee}{\end{equation}}
\newcommand{\ba}{\begin{eqnarray}}
\newcommand{\ea}{\end{eqnarray}}
\newcommand{\baa}{\begin{align}}
\newcommand{\eaa}{\end{align}}
\newcommand{\nn}{\notag}
\newcommand{\qq}{\qquad}
\newcommand{\lb}{\label}
\newcommand{\mat}[1]{\begin{pmatrix} #1\end{pmatrix}}

% notation for the operators
\newcommand{\op}[1]{\hat {#1}}
\newcommand{\sop}[1]{\op{\op {#1}}}
\newcommand{\commutator}[2]{\left[ {#1} , {#2} \right]}
\newcommand{\trace}[1]{\mathrm{tr}\left(#1\right)}
\newcommand{\argument}[1]{\ifthenelse{\isempty{#1}{}}{}{(#1)}}
\newcommand{\matop}[1]{\mathbf{#1}}
\newcommand{\optr}[1]{\check #1}
\newcommand{\opskew}[1]{{\op {#1}}_{\perp}}

% notation for the states
\newcommand{\tket}[1]{| \tilde #1 \rangle}
\newcommand{\tbra}[1]{\langle \tilde #1 |}
\newcommand{\brket}[2]{\langle  #1 | #2 \rangle} %standard braket
\newcommand{\tbraket}[2]{\langle \tilde #1 | #2 \rangle}
\newcommand{\ketbra}[2]{| #1 \rangle \langle #2 |}
\newcommand{\tketbra}[2]{| #1 \rangle \langle \tilde #2 |}
\newcommand{\sket}[2]{| #2)^{#1}}
\newcommand{\sbra}[2]{( #2|_{#1}}
\newcommand{\sketor}[2]{| #2]^{#1}}
\newcommand{\sbraor}[2]{[ #2|_{#1}}
\newcommand{\sbraket}[2]{\braket{\op{#1} | \op{#2}}}
\newcommand{\dket}[1]{\Bigl| #1 \Bigr)}
\newcommand{\dbra}[1]{\Bigl(#1 \Bigr|}
\newcommand{\dbraket}[2]{\Bigl(#1 \Bigl| #2 \Bigr)}
\newcommand{\dketor}[1]{\Bigl| #1 \Bigr]}
\newcommand{\dbraor}[1]{\Bigl[#1 \Bigr|}
\newcommand{\hket}[1]{| #1 ]}
\newcommand{\hbra}[1]{[ #1 |}
\newcommand{\hbraket}[2]{[#1 | #2 ]}

% special operators
\newcommand{\dmnot}{\op{\rho}_0}
\newcommand{\dm}{\op{\rho}}
\newcommand{\hnot}{\op{H}_0}
\newcommand{\hone}[1]{\op{H}_1\argument{#1}}
\newcommand{\transition}[1]{\op T_{#1}}
\newcommand{\excite}[2]{\op e_{{#1}{#2}}}
\newcommand{\decay}[2]{\op d_{{#1}{#2}}}
\newcommand{\Liouv}{\sop{\mathcal L}}
\newcommand{\Liouvnot}{\sop{\mathcal L_0}}
\newcommand{\coupl}{\sop{\mathcal K}}
\newcommand{\honetmp}[1][]{\op{H_1}\argument{#1}} 
\newcommand{\identity}{\op{\mathbb I}}
\newcommand{\rmat}[1]{\optr R}


\begin{document}

%\preprint{APS/123-QED}

\title{On the analytic structure of the linear response susceptibility in open systems}

\author{Luigi Genovese}
\affiliation{Laboratoire de Simulation Atomistique (LSim), SP2M, INAC, CEA-UJF, 17 Av. des Martyrs,
38054 Grenoble, France}
\author{Marco D'Alessandro}%
\affiliation{Istituto di Struttura della Materia-CNR (ISM-CNR), Via del Fosso del Cavaliere 100, 00133 Roma, Italia}

\date{\today}

\begin{abstract}
By performing linear-response (LR)-TDDFT calculation of molecular systems within a highly complete real-space basis set, we analyze how the behavior of the dynamical 
polarizability is influenced by the discretization conditions of the computational treatment.
We show that optical excitations of energy below the LR ionization potential behave as observable, localized quantities, that can be studied with high precision, 
provided an adequate level of completeness. We then present indicators that can help to quantify such potential observable property of an excitation, that can be evaluated 
in any discretization scheme. Under this light, we also show that excitation energies above ionization threshold do not exhibit such observable features and \emph{cannot} be 
considered as poles of the dynamical polarizability. This result is a inherent behavior of the system's Liouvillian, and does not depend on computational treatment of the 
unoccupied subspace.
\end{abstract}

%\pacs{Valid PACS appear here}
\maketitle

\section{Introduction}

.....

\section{Theory}

In the framework of the linear response formalism one is typically interested in computing the \emph{linear response functional}:
\be\lb{LinearResponseFunctDef1}
{\cal R}_{\op O}(\omega) = \trace{\dm'(\omega)\op O} \;,
\ee
that represents the perturbation-induced modification of the expectation value of an observable $\op O$\footnote{For a sake of concreteness, we are limiting our considerations 
to observables that do not explicitly depend on time}. Here $\dm'(\omega)$ is the \emph{response density operator} that describes the $\omega$-dependent modification of the density 
operator induced by the perturbation.

\subsection{Basic equation for the linear response formalism for electronic excitations}

The linear response formalism can be effectively applied to the analysis of the electronic excitations of molecular and condensed-matter systems, described by the (density-dependent) 
ground-state hermitian Hamiltonian $\hnot$, coupled to an external perturbation described by the perturbing field $\op\Phi(\omega)$. The $\omega$ structure of $\op\Phi(\omega)$ codifies both 
the intrinsic time dependence of the perturbation and the switching-on procedure, \emph{e.g.} adiabatic or sudden limit protocols. 

The response density is expressed as the first-order variation of ground state density
\be\label{rhodef1}
\dmnot = \sum_{\{p\}} \ket{\psi_p} \bra{\psi_p}  \;,
\ee
where $\{\psi_p\}$ denote the occupied states, \emph{i.e} the subset of the bound eigenstates of $\hnot$ with energy lower than the Fermi level.
By considering the linear order time evolution of these states in presence of the perturbation we can express the response density in the $\omega$ domain as follows:
\be\lb{rhoPrimeOmegaDef1}
\dm'(\omega) = \ii \commutator{\dmnot}{\op A(\omega)} \;,
\ee
where $\op A(\omega)$ is a sum of two terms, both written as a triple convolution with the retarded/advanced Green functions $\op G_0^{\pm}$ of the unperturbed Hamiltonian:
\be
\op A(\omega) = \frac{1}{2\pi} \int_{-\infty}^\infty  \dd \omega'
\op G_0^+(\omega') \left(\op \Phi(\omega)+\op V'(\omega)\right) \op G_0^-(\omega'-\omega)\;. \nn
\ee
The presence of the second addend is due to the density dependence of the unperturbed Hamiltonian and describes the perturbation induced by the variation of $\hnot$ under the
action of the perturbing field. In the linear response regime this quantity can be expressed as the functional derivatives of $\hnot$ with respect to the density kernel 
(evaluated at $\dm=\dmnot$) times the value of the induced density modification, that is:
\be\label{VpDef1}
\op V'(\omega)= 
\int \dd \r \dd \r' \frac{\delta\hnot}{\delta \rho_0(\r,\r')}
\trace{\dm'(\omega) \ket{\r} \bra{\r'}} \;.
\ee

\subsection{Fluctuation states. A localization argument for the linear response functional}

We start this analysis by deriving an ulterior expression of the response density operator in the frequency space. We consider equation \eqref{rhoPrimeOmegaDef1} and observe that, for any 
operator $\op O$, we may write:  
\be
\commutator{\dmnot}{\op O} = \dmnot \op O \op Q_0 - \op Q_0 \op O \dmnot\;,
\ee
where $\op Q_0= \identity - \dmnot$ is the projection operator in the subspace orthogonal to $\dmnot$. So, thanks to this relation we have that:
\be
\dm'(\omega) = \ii\sum_p\left(\ketbra{\psi_p}{\psi_p}\op A(\omega)\op Q_0 - \op Q_0\op A(\omega)\ketbra{\psi_p}{\psi_p}\right) \;.
\ee
This equation suggests us to define the \emph{fluctuation states}:
\be\lb{fluctuationStateDef1}
\ket{f_p(\omega)} \doteq -\ii\op Q_0\op A(\omega)\ket{\psi_p} \;, \quad
\bra{f_p(\omega)} \doteq \bra{\psi_p}\ii\op A(-\omega)\op Q_0 \;,
\ee
where the $\omega$ dependence of the bra is consistent with hermitian conjugation property of the Fourier transform of a hermitian operator. Fluctuation states are, by 
definition, orthogonal to the occupied eigenstates of $\hnot$, so that:
\be
\brket{f_p(\omega)}{\psi_q} = \brket{\psi_q}{f_p(\omega)} = 0 \;, \forall \, p,q,\omega \;.
\ee
The response density written in terms of the fluctuation states reads:
\be\lb{rhoPrimeFluctuationStateDef1}
\dm'(\omega) = \sum_p\left(\ketbra{\psi_p}{f_p(-\omega)} + \ketbra{f_p(\omega)}{\psi_p}\right) \;,
\ee
and consequently the linear response functional can be expressed as:
\be\lb{LinearResponseFunctDef2}
{\cal R}_{\op O}(\omega) = 
\sum_p\left(\bra{f_p(-\omega)}\op O\ket{\psi_p} + \bra{\psi_p}\op O\ket{f_p(\omega)}\right) \;.
\ee



\bibliography{Analytic_biblio}

\end{document}
